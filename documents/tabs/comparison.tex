\begin{table}
\centering
\begin{center}
\caption{Power law slopes from different composites.}
\tablabel{comparison}
\begin{tabular}{cc}
\hline
\noalign{\smallskip}
Reference &  FUV slope, $\alpha$$^{(a)}$ \\  
\hline


This work  & $-1.70$   \\
Lusso et al. 2015  & $-1.39$   \\
Telfer et al. 2002  & $-1.31$   \\
Francis et al. 1991  & $-1.68 $   \\

Vanden Berk et al. 2001  & $-1.56$   \\
Glikman et al. 2006 & $-(0.45 - 1.63)$ $^{(b)}$  \\
%Composite fit slope geo...-1.72090273018 +- 3.94825826696e-06
%Composite fit slope wmean...-1.6990493655 +- 8.02966649592e-08
%Composite fit slope mean...-1.73597882263 +- 2.96733603083e-06
%Composite fit slope median...-1.72180526708 +- 5.81882905153e-05
%Individual slope mean...-1.71028816373 +- 0.0884323626711
%Individual slope median...-1.68684199155 +- 0.0884323626711
\hline
\hline
\end{tabular}
\end{center}
\noindent{
$^{(a)}$ The slope of a power law in the Far UV.
$^{(b)}$ The range of slopes is due to different combination methods and regions selected for a power law fit. 
}


\end{table}



