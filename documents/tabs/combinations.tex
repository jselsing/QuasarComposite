\begin{table}
\centering
\begin{center}
\caption{Power law slopes from different combination methods. \label{tab:targets}}

\begin{tabular}{cc}
\hline
\noalign{\smallskip}
Combination method &  $\alpha$$^{(a)}$ \\  
\hline


Weighted arithmetic mean  & $-1.699\pm 8 \cdot 10^{-8}$   \\
Arithmetic mean  & $-1.736\pm 3 \cdot 10^{-6}$   \\
Geometric mean  & $-1.721\pm 4 \cdot 10^{-6}$   \\
Median  & $-1.722\pm 6 \cdot 10^{-5}$   \\

Individual mean$^{(b)}$  & $-1.71\pm 0.09$   \\
Individual median$^{(c)}$ & $-1.69$   \\
%Composite fit slope geo...-1.72090273018 +- 3.94825826696e-06
%Composite fit slope wmean...-1.6990493655 +- 8.02966649592e-08
%Composite fit slope mean...-1.73597882263 +- 2.96733603083e-06
%Composite fit slope median...-1.72180526708 +- 5.81882905153e-05
%Individual slope mean...-1.71028816373 +- 0.0884323626711
%Individual slope median...-1.68684199155 +- 0.0884323626711
\hline
\hline
\end{tabular}
\end{center}
\noindent{
$^{(a)}$ The slope of a power law.
$^{(b)}$ The arithmetic mean of the slopes fitted in the individual spectra. The reported $1-\sigma$ errors are the standard deviation of the individual slopes.
$^{(c)}$ The median value of the individual slopes. No reasonable error can be calculated due to small sample size. 

}


\end{table}



